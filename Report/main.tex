\documentclass[conference]{IEEEtran}

\usepackage[utf8]{inputenc}
\usepackage{amsmath,amssymb}
\usepackage{graphicx}
\usepackage{cite}



\begin{document}

\title{Quantum Error Correction (EE7001)}

\author{
    \IEEEauthorblockN{Tanay Bhat}
    \IEEEauthorblockA{
        Department of Electrical Engineering \\
        Indian Institute of Technology, Bombay \\
        Mumbai, India \\
        Email: 22b3303@iitb.ac.in
    }
    \and
    \IEEEauthorblockN{Jay Mehta}
    \IEEEauthorblockA{
        Department of Electrical Engineering \\
        Indian Institute of Technology, Bombay \\
        Mumbai, India \\
        Email: 22b1281@iitb.ac.in
    }
    \and
    \IEEEauthorblockN{Rajwardhan Toraskar}
    \IEEEauthorblockA{
        Department of Electrical Engineering \\
        Indian Institute of Technology, Bombay \\
        Mumbai, India \\
        Email: 22b0721@iitb.ac.in
    }
}

\maketitle

\begin{abstract}
This is the abstract section.
\end{abstract}

\section{Introduction}
This is \cite{sample2023} the introduction.

\section{Paper I - QEC for Beginners}
This section deals with the theory presented in the first paper\cite{paper1}.

\section{Paper II - Entanglement Purification and QEC}
This section deals with the theory presented in the first paper\cite{paper2}.


\section{Implementation and Experimental Results}
This section deals with the implementation and experimental results.

\section{Conclusion}
This is the conclusion.

\bibliographystyle{IEEEtran}
\bibliography{refs.bib}

\end{document}
